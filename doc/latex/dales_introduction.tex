\introduction

Modern atmospheric research typically relies on a cascade of observational and modeling tools. The largest-scale models, such as the General Circulation Models (GCM), contain parameterizations that were developed with the help of observational campaigns, but more and more also with the help of limited area modeling. With respect to the atmospheric boundary layer (ABL), Large-Eddy Simulations (LES) are arguably the most detailed type of numerical modeling available. 

The principle of LES is to resolve the turbulent scales larger than a certain filter width, and to parameterize the smaller, less energetic scales. This filter width is usually related to the grid size of the LES, and ranges typically between $1\mr{m}$ for stably stratified boundary layers, to $50\mr{m}$ for simulations of the cloud-topped ABL. In such a typical LES set up, up to $90\%$ of the turbulent energy resides in the resolved scales.  In the fields where LES is applicable, it has the advantage over coarser models that LES relies only weakly on parameterizations. In comparison with observational studies, LES has the advantage of providing a complete data set, in terms of time, space, and in terms of diagnosable variables. Especially the combined use of LES and observations is a popular methodology in process studies of the ABL.

LES modeling  of the ABL started in the late sixties \cite[e.g.,][]{lilly1967,deardorff1972}; cloudy boundary layers were first  simulated by \citet{sommeria1976}.  From \citet{nieuwstadt1986} onward, several cycles of intercomparison studies compare state-of-the-art LES models with observational studies and with each other. The aim of these studies was not so much to determine which LES model performs best in which situation, but more to determine where the general strong points of atmospheric LES lies, and in what fields LES still has room for improvement. Two particularly active cycles are organised under the umbrella of the Global Energy and Water Cycle Experiment (GEWEX): the GEWEX Atmospheric Boundary Layers Study (GABLS), and the GEWEX Cloud System Study (GCSS) Boundary Layer Cloud Working Group. The GABLS focusses on the clear boundary layer, mainly on stable and transitional situations \citep{holtslag2006,beare2006,basu2008}. The GCSS looks at different aspects of boundary layer clouds, mainly shallow cumulus and stratocumulus clouds \citep{bretherton1999,bretherton1999b,duynkerke1999,duynkerke2004,brown2002,siebesma2003a,stevens2001,stevens2005,ackerman2009,zanteninpreparation}. 

The Dutch Atmospheric Large-Eddy Simulation (DALES) has joined in virtually all of these intercomparisons. Besides convective, stable and cloud-topped boundary layers, DALES has also been used on a wide range of topics, such as studies of shear driven flow, heterogeneous surfaces,  dispersion and turbulent reacting flows in the ABL, and of flow over sloped terrain. As such, DALES is one of the most all-round tested available LES models for studies of the ABL. In this paper, we aim to describe and validate DALES3.2, the current version of DALES.

In the remainder of this paper, we first give a thorough description of the code in \secnm \ref{sec:dales_description}. In \secnm \ref{sec:dales_use}, an overview of studies condcuted with DALES are given, both as a validation of the code as well as an overview of the capabilities of an LES like DALES. In \secnm \ref{sec:dales_outlook}, an outlook is given on future studies that are planned to be done DALES, as well as an outlook on future improvements.
